In this thesis we achieve multiple goals. First we show that rankings of financial analysts are important to investors as a trading strategy based on the rankings outperforms the ones that are based on the consensus or a buy and hold strategy. Second, we adapt the label ranking algorithm to map the rankings with the variables that affect these rankings (state variables) and, further, we identified a set of variables that affect the rankings the most. Finally, we successfully predict the rankings and we show that a strategy based on the predicted rankings outperforms the baselines. 

Our work has the following limitations. First, our ranking model of financial analysts does not take into consideration analysts' verbal recommendations about a stock. In our work we assume that by setting an upward (downward) price target an analyst signals a ``buy" (``sell") recommendation for investors. It would be interesting to incorporate all three components of analysts' report (explicit stock recommendations, EPS forecasts, and price targets)  into a single ranking and apply models developed in this thesis.  Second, we select a naive Bayes label ranking algorithm to predict the rankings mainly due to the Bayesian framework used in  the Black-Litterman model. For the future work, we may extend the analysis to other label raking models, such as the nearest neighbor or association rules. Finally, our work could be applied in different ranking domains; for example, predicting the rankings of mutual funds based on their annual performance. 

