\chapter*{Resumo}
Os rankings dos analistas financeiros são amplamente utilizados por fornecedores de informação financeira para avaliar o desempenho dos analistas em termos das suas recomendações de investimento, preços-alvo, e estimativas de resultados. Por um lado, os analistas no topo do ranking recebem maior reconhecimento entre os seus pares e têm mais atenção de potenciais investidores. Por outro lado, os fornecedores da informação financeira utilizam os rankings para criar recomendações ``inteligentes'' e vendê-las aos investidores. Assim, os rankings dos analistas são importantes e relevantes para muitos participantes no mercado. Neste trabalho mostramos que é possível prever os rankings  e usá-los para delinear uma estratégia de investimento que resulta em retornos ajustados pelo risco positivos acima dos que seriam gerados por uma estratégia de investimento no mercado ou baseada em estimativas de consenso. O trabalho realizado inclui ainda a  caracterização do desempenho relativo dos analistas; e a identificação das variáveis ​​que mais contribuíram para alterações  nos rankings. Para responder a estas questões, adaptámos e aplicámos um algoritmo de  Machine Learning que modeliza a relação entre variáveis económicas ​​e os rankings. O algoritmo baseia-se no conceito de semelhança entre os rankings e utiliza uma probabilidade de ``label ranking'', obtido a partir da transformação Bayesiana, para prever o ranking mais provável dado um conjunto das variáveis ​​independentes. Em resumo, a contribuição de nossa trabalho é a seguinte: primeiro, mostramos que as estratégias de investimento baseadas nos rankings dos analistas superam aquelas baseadas mas estimativas de consenso dos analistas; segundo, adaptamos um algoritmo de classificação para resolver um problema de previsão dos rankings; terceiro, analisamos a capacidade explicativa das variáveis económicas ​​de estado e identificamos as que determinam alterações nos rankings; finalmente, conseguimos prever os rankings dos analistas  e mostramos que as estratégias de investimento baseadas em previsões de rankings superam aqueles baseados em rankings previstas sem modelo.
