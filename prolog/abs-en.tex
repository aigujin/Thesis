\chapter*{Abstract}
Rankings of financial analysts are widely used by  financial research vendors to evaluate the analysts' performance in terms of their recommendations, price target accuracy, and earnings forecasts accuracy. On one hand, top analysts receive a greater acknowledgment among their peers and more attention from the potential investors. On the other hand, financial information vendors utilize the rankings to create smart recommendations  and sell them to investors. Thus, rankings of financial analysts are important and relevant for many financial market players. In this work we show that it is possible to predict rankings of financial analysts and use these rankings in active trading strategies with risk-adjusted returns above market returns and those that would result from using consensus estimates. We also address the problem of characterizing the general behavior of analysts' relative performance and identifying the variables that contributed the most to changes in rankings.  We solve these tasks by adapting and applying a Machine Learning label ranking algorithm that models a relation between the state variables and the rankings. The algorithm relies on the concept of ranking similarity and uses a label ranking probability, obtained from the Bayesian transformation, to predict the most probable ranking given a set of descriptive independent variables.  In summary, the contribution of our research is four-fold. First, we show that trading strategies based on analysts' rankings outperform those based on analysts' consensus estimates; second, we adapt a \emph{naive Bayes} classification algorithm to solve a label ranking problem; third, we analyze  state variables and identify the most contributive to changes in rankings; finally, we predict the rankings of financial analysts and show that the  trading strategies based on model-predicted rankings outperform those base on a non-model rankings.
