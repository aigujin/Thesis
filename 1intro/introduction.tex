%Rankings of financial analysts is not new in finance. Many agencies develop their procedures to evaluate analysts based on their performance either in forecasting or stock recommendations. Some institutions even hold a ``Red Carpet'' event to recognize the top analysts. 

In recent years, some institutions have been very active in publishing and  selling  rankings of financial analysts. Some  are based on privately held surveys of  buy-side analysts (e.g., the Institutional Investor's rankings of the All-America Research Teams\footnote{\url{http://www.institutionalinvestor.com/Research/4560/First-Team.html}} and Bloomberg's America's Best Stock Analysts\footnote{\sloppy \url{http://www.bloomberg.com/news/2013-08-14/jpmorgan-top-stock-picker-with-equities-out-of-lockstep.html}}); others are based on the performance of sell-side analysts (ThomsonReuters' top StarMine analysts\footnote{\url{http://excellence.thomsonreuters.com/award/starmine}}). In any of these cases, the rankings aim at identifying the top analysts. However, aside from personal acknowledgment among peers, it is still arguable whether these are useful to investors~\citep{desai2000ass} or are merely ``popularity contests''~\citep{emery2009}. 


%The idea that financial analysts play an important role in financial markets is rather consensual~\citep{cowles1933csm,obrien1990}. Yet there is some debate on whether following the advice of analysts brings value to investors after transaction costs \citep{womack1996,mikhail2004sae,li2005persistence}. Related to this is the difficulty in identifying the analysts with superior stock picking skills. 


In this thesis we show that rankings of financial analysts are useful to investors because strategies based upon these rankings yield positive abnormal returns and that the top ranked analysts have stock picking skills. We combine two areas of research: Computer Science and Financial Economics in order to solve  problem of predicting the rankings and applying them into trading strategies. From the Computer Science we adapt a Machine Learning algorithm that we use to predict the rankings and from Financial Economics, we select a model to blend these rankings with a set of variables that affect the analysts' decision in issuing stock related forecasts. 



%The contributions of our research is fourfold. First, we develop a trading strategy that transforms the rankings of financial analysts into inputs for the Black-Litterman model~\citep{black1992}. Second, we show that annualized cumulative returns generated by some trading strategies  based upon analysts' rankings outperform a passive strategy (e.g., buy-and-hold the general stock market index). Third, we show that the strategy based upon the perfect foresight of rankings yields the highest cumulative annualized return. Fourth, we find that investors  are better off following analysts that issue the most accurate target prices, rather than those that issue the most accurate EPS forecasts.


Label ranking is an increasingly popular topic in the machine learning literature. It studies the problem of learning a mapping from instances to rankings over a finite number of predefined labels. In some sense, it is a variation of the conventional classification problem; however, in contrast to the classification settings, where the objective is to assign examples to a specific class, in label ranking we are interested in assigning a complete preference order of labels to every example~\citep{cheng2009}.

Many algorithms have been adapted to deal with label ranking such as: decision-trees for label ranking~\citep{cheng2009}, algorithm based on Plackett-Luce model~\citep{cheng2010}, pairwise comparison~\citep{hullermeier}, and k-NN for label ranking~\citep{brazdil2003}. \ref{ch2:lr-summary} outlines the recent developments in solving a label ranking problem.



The \gls{emh}~\citep{fama1970ecm} suggests that all public information available to investors is incorporated in prices and new information is immediately reflected in valuations. Yet there are information gathering costs and financial analysts are better than an average investor at processing this information which reflects in issued buy/sell recommendations. These recommendations, like other news about the general economy as well as about a particular company, influence investors' perception and beliefs.

Previous studies show that analysts stock recommendations have investment value ~\citep{womack1996,barber2001}. The literature also suggests further that foreknowledge of analyst forecast accuracy is valuable~\citep{brown2003,aiguzhinov2015a}. In line with academic research findings, practitioners too pay attention to analyst forecast accuracy rankings. On an annual basis, firms such as The Institutional Investor and StarMine\footnote{http://www.starmine.com} publish analysts ratings according to how well they performed, based partly on past earnings forecast accuracy.

The importance of these ratings should not be ignored because the attention that the market gives to the recommendations of different analysts is expected to correlate with them. Typically, the performance of analysts is analyzed in terms of their individual characteristics (e.g., experience, background)~\citep{clement1999}. The disadvantage of this approach is that the collection of the necessary data is difficult and it is not always reliable. As for practitioners, they rely mostly on past accuracy to predict future accuracy.




In general, rankings of financial analysts are widely used by  financial research vendors to evaluate the analysts' performance in terms of their recommendations, price target accuracy, and earnings forecasts accuracy. On one hand, top analysts receive a greater acknowledgment among their peers and more attention from the potential investors. On the other hand, financial information vendors utilize the rankings to create smart recommendations  and sell them to investors. Thus, rankings of financial analysts are important and relevant for many financial market players. 
 

 
The thesis is organized as follows: \ref{ch1} investigates whether rankings of financial analysts could be  beneficial to investors. Indeed, the financial literature still debates on whether following advice of financial analysts brings value to investors after transaction costs. If it does, then it is worth to try  to identify the analysts with the superior stock picking skills. We show that by ranking the analysts on the basis of their accuracy in forecasting  stock prices in one year horizon, we are able to develop trading strategies that generate  positive abnormal returns.  We develop a trading strategy that transforms the rankings of financial analysts into inputs for the \gls{blm}~\cite{black1992}. The model incorporates ``views'' in a \gls{capm} framework, forming optimal portfolios in a mean-variance optimization setting. ``Views'' are expectations on individual stocks’ future performance obtained from the weighting an individual analyst’s according to her rank: the view of the analyst that has a rank of 1 is weighted by 100\% and the further ranks down, the lesser is the weight. In this setup, we assume and further test if top ranked analysts have stock picking skills. To obtain the information for the ranking, we look at three possibilities. First, we look at last known rank of analysts; this way, we see if the \naive{} development in the stock market contributes to the analysts’ knowledge. Second assumption is that the \default{} history of a stock is important for a stock valuation analysis. Third, is a hypothetical assumption where we use a perfect forecast and assume investors have perfect foresight of analysts' rankings. Each of these three settings resulted in three different sets of rankings that we use as inputs for the Black-Litterman model. The strategy based on the rankings with the perfect foresight yields of course the maximum annualized cumulative returns. Out of the feasible strategies, a strategy based on rankings of analysts who issue more accurate price targets and use the most recent information set outperforms all other feasible strategies. The evidence confirms our assumption that rankings are important and further proves that by obtaining the recent rankings and setting up a simple trading strategy, an investor gains a cumulative annualized return that is higher than that of a passive strategy of buy-and-holding the market. These results suggest further that investor should try to predict the rankings of financial analysts.

\ref{ch2} explores the methodology  of predicting the rankings of items (labels) and introduces a Machine Learning algorithm for the purpose of predicting the analysts' rankings. The main research questions are: given observed rankings and independent variables that describe the differences in these rankings, what would be the next most likely (similar) ranking? To answer this question, we adapt the naive Bayes classification algorithm to deal with this problem. Despite its limitations, this algorithm demonstrates good results in many applications. Specifically, the Bayesian framework is well understood in many domains such as Financial Economics where Bayesian models are widely used (e.g., the Black-Litterman model for active portfolio management). An algorithm models, via calculating conditional probabilities, a mapping between independent variables and corresponding rankings and, once a new set of independent variables is observed, it predicts the ranking. The paper shows the results of the experiments in testing the performance of the algorithm compared to other label ranking algorithms. We show that it consistently outperforms a baseline method and is competitive with other algorithms.

\ref{ch3} analyzes the variables that drive the differences in analysts’ rankings. We investigate whether it is possible to identify factors that influence analysts’ opinions by looking at states in which these analysts make decisions as opposed to analyze characteristics of individual analysts as done in previous literature. To achieve this goal, we first, build rankings of analyst based on their \gls{eps} forecasts accuracy. Then, we select the state variables that are responsible for the  differences of analysts’ ranks. Finally, we apply a Machine Learning label ranking algorithm to build a model that relates the rankings with the variables and calculates a discriminative power of a variable, i.e., the contribution of each variable to rankings. Our results suggest that the variation in rankings is due to the different ability of the analysts to interpret the information environment (e.g., whether the market is bull or bear). We, thus, select and analyze variables that characterize this environment. To capture the full spectrum of the analyst’s decision making process, we select variables based on different levels of information availability: analyst-specific, firm-specific and about the  general economy. For each level, we choose variables  that  information asymmetry and uncertainty. As for analyst-specific variables, we select \emph{dispersion}, \emph{asymmetry}, and \emph{uncertainty} in forecast. On a stock-specific level, we use the following variables: \emph{book-to-market}, \emph{debt-to-equity}, \emph{size}, \emph{stock returns}, \emph{accruals}, and \emph{industry returns}. Finally, to characterize the general economy, we select \emph{GNP}, \emph{inflation}, \emph{interest rate}, and general \emph{market volatility}. We also control for different states of the variables across time and introduce three methods to capture it: the first-difference of variables, the random part of the time-series decomposition, and the rolling eight-quarter standard deviation. We report that, controlling for the dynamics in the variables, the variables that influence the rankings are particularly those that capture the conditions of  the general economy. For the first difference method, the variable with the most discriminative power is the \emph{interest rate}; for the random dynamic state, the most significant variables are \emph{GNP} and \emph{inflation}. Finally, for the rolling standard deviation, it is again \emph{interest rate}. For the static state of the variables, the most significant variable is the \emph{dispersion} in EPS  forecasts  among analysts. Given the identified state variables we can use them to predict the rankings of financial analyst applying label ranking algorithm. 

In \ref{ch4} we apply the selected discriminative variables and predict the analysts' rankings. We report that, for the case when analysts' rankings are based on the price target accuracy,  our model can predicted rankings that are more  accurate than those obtained from the simple model of using the \default{} average analysts' rankings. For the case of rankings based on the \gls{eps} forecasts accuracy, our model is able to predict rankings that are better than those of the \default{} and the \naive{} baselines. Moreover, the supremacy of our model occurs when  the variables that characterize analysts' information environment exhibit the first-difference dynamic state. We also perform a back-test of active trading using the predicted rankings as inputs for the Black-Litterman model. The results show that the strategies based on the analysts' rankings outperform, in terms of the annualized cumulative return, a  strategy based on the analysts' consensus. Furthermore, out of the ranking based trading strategies, the maximum annualized cumulative return yields  a strategy that is based on the predicted rankings with the first-difference of state variables.  Thus, we show that based on the predictions of the rankings of financial analysts, it is possible to devise  successful trading strategies.

Finally, \ref{conclusion} summarizes the main results of the thesis, outlines its limitations, and proposes the ideas for the future work. 
